% Choose document class type
\documentclass[12pt]{article}
 
% Import packages
\usepackage[margin=1in]{geometry}
\usepackage{graphicx}
\usepackage{titling}
\usepackage{multicol}
\usepackage{float}
\usepackage[font={it}]{caption}
\graphicspath{ {/Users/malenarice/Desktop/Astro120/Labs/Lab1/Images/} } % Path to folder where your images are stored

% Produce a title and header
\setlength{\droptitle}{4em} 
\begin{document} 
\title{\vspace{-2cm} Title of my document \vspace{-0.6cm} }
\author{By: Malena Rice \\ Group Members: Imad Pasha, Gaspard Duchene\thanks{Additional comments and acknowledgements.}}
\date{}
\maketitle

% Use vspace to make your document more compact
\vspace{-0.5cm}

% Produce 2 columns to make the document compact
\begin{multicols}{2}

% Create titles for each section; use \section*{} to get rid of the section number
\section{Abstract}
CONCISE description of your lab and major results.

\section{Introduction}
Introduce the experiment.

\section{Equipment and Data}
Discuss data acquisition and describe equipment.

\section{Analysis}
Analyze the retrieved data.

% Produce a table
\vspace{2mm}
\begin{table}[H]
\begin{center}
\begin{tabular}{| c | c |}
\hline
\textbf{Column 1} & \textbf{Column 2} \\ \hline % Bold title names
Element 1 & Value 1 \\ \hline
Element 2 & Value 2  \\ \hline
\end{tabular}
\caption{Insert Your Caption Here.}
\label{tab:size_categories}
\end{center}
\end{table}

% Reference a figure; figure number will auto-update as you add figures.
Refer to figures in the text using references: for example, Figure \ref{fig:my_plot}. 

% Produce a figure
\begin{figure}[H]
\centering
\includegraphics[scale=0.4]{figure_1.png}
\caption{Caption your figure here.}
\label{fig:my_plot}
\end{figure}

% Produce an equation
\begin{equation}
F = \sigma_{SB} T^{4}
\label{eq:my_equation}
\end{equation}

% Math references outside of an equation use Math Mode. Use $ to enter/exit math mode:
Math mode example: $\alpha \beta \gamma$

One full line in math mode:
$$ F = \frac{L}{4\pi d^2} $$

% Producing a numbered list:
\textit{Numbered list:} % Italicize with \textit{}

\begin{enumerate}
\item{First item}
\begin{enumerate}
\item{Nested item 1}
\item{Nested item 2}
\end{enumerate}
\item{Second item}
\end{enumerate}

% Producing a bullet point list:
\textit{Bullet point list:} 

\begin{itemize}
\item{First item}
\item{Second item}
\begin{itemize}
\item{Nested item 1}
\item {Nested item 2}
\end{itemize}
\end{itemize}

% Remember to end your multicols and document! LaTeX will complain if you forget these lines.
\end{multicols}
\end{document}